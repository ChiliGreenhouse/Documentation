\section{Einblick in den Backend-Quellcode}
    Dieses Kapitel gibt einen Einblick in den Quellcode des Backends.
    Dazu wird ein bestimmer REST-Endpunkt betrachtet:

    $$
        \text{\lstinline{POST /api/v0/gh/0/devices/[device_name]/datapoints}}
    $$
    Er ist dafür verantwortlich, einen neuen Sensor-Datenpunkt aufzunehmen, wird also regelmäßig von der Hardware aufgerufen.
    Der URL-Pfad ist zusammengesetzt aus vier Teilen:
    \begin{itemize}
        \item \lstinline{/api/v0}: Versionierung der API.
        \item \lstinline{/gh/0}: Es wird das Greenhouse (gh) mit Index 0 angesprochen\footnote{Dies ist ein Platzhalter für den zukünfigten Multi-Greenhouse Betrieb.}.
        \item \lstinline{/devices/[device_name]}: Es wird das Sensor-Device mit gegebenem Namen betrachtet.
        \item \lstinline{/datapoints}: In Verbindung mit dem HTTP Operator \lstinline{POST} heißt dies, dass ein neuer Datenpunkt erstellt werden soll.
    \end{itemize}
    Hier hat allerdings nur der Teilpfad \lstinline{/devices/[device_name]/datapoints} Relevanz.
    Er beinhaltet ein Argument des REST-Endpunktes: Den Namen des Sensor-Devices.

    In Kapitel~\ref{sec:methodik} wurde bereits die Schichtenarchitektur des Backends präsentiert.
    Diese spiegelt sich im Quellcode gut sichtbar wieder.
    Zuerst wird die nötige Funktion in der Datenbankschicht betrachtet.
    Ihr Rust-Code ist in Abbildung~\ref{lst:database-code} zu sehen.
    In Zeile 1-5 befindet sich der Funktionskopf, welche als Parameter die Datenbankverbindung (conn), den neuen Sensor-Datenpunkt und den Sensor-Devicenamen definiert.
    Nach einer ersten Typ-Umwandlung folgt der Bau des \lstinline{INSERT} SQL-Queries mithilfe der Rust-Bibliothek \lstinline{diesel}\footnote{siehe https://diesel.rs/}.
    \lstinline{Diesel} ist ein ORM (object-relational mapping) tool, um u.a. SQL-Queries direkt aus Rust-Code zu bauen.
    Einige Vorteile sind die Typ-Sicherheit oder ein automatischer Schutz vor SQL-Injections.
    Zuletzt wird dann, falls der Datenbankaufruf erfolgreich war, der neue Sensor-Datenpunkt zurückgegeben.

    \begin{figure}[H]
        \centering
        \begin{minted}[linenos]{Rust}
pub fn create_datapoint(
    conn: &mut PgConnection,
    datapoint: SensorDatapoint,
    sensor_name: String,
) -> Result<SensorDatapoint, DatabaseError> {
    // Umwandlung zu Datenbank-struct
    let db_datapoint = DbSensorDatapoint::from_model(
        datapoint,
        sensor_name
    );

    // Bau und Ausführung eines SQL-Queries mit diesel
    // Der ?-Operator bricht die Funktion bei einem Fehler ab.
    let db_datapoint = db_datapoint
        .insert_into(super::schema::sensor_datapoints::table)
        .returning(DbSensorDatapoint::as_returning())
        .get_result(conn)?; 

    // Rückgabe des neuen Sensor-Datenpunkes
    Ok(db_datapoint.into())
}
        \end{minted}
        \caption{Rust-Funktion der Datenbankschicht zur Erstellung eines neuen Sensor-Datenpunktes}
        \label{lst:database-code}
    \end{figure}

    Als nächstes wird die Funktion der Serviceschicht vorgestellt, welche den zuvor genannten REST-Endpunkt bereitstellt.
    Ihr Quellcode ist in Abbildung~\ref{lst:endpoint-code} dargestellt.
    Der Funktionskopf von Zeile 1 - 5 ist hier von besonderer Bedeutung.
    Die Rust-Bibliothek \lstinline{axum}\footnote{siehe https://github.com/tokio-rs/axum} bietet eine Art Framework für die Erstellung von HTTP-Servern.
    Sie ist auch dafür verantwortlich, dass die Funktionsparameter richtig gesetzt werden.
    Es werden von ihr also die Datenbankverbindung (Zeile 2), der Sensor-Name aus dem URL-Pfad (Zeile 3) und die übermittelten JSON-Daten (Zeile 4) überliefert.

    Die Funktion stellt dann die Datenbankverbindung her (Zeile 7), überprüft mit einer Hilfsfunktion ob es sich bei dem angegebenen um ein valides Sensor-Device handelt (Zeile 10) und verpackt erstellt ein neues SensorDatapoint-struct (Zeile 13-16).
    Dann wird die zuvor vorgestellte Funktion der Datenbankschicht aufgerufen (Zeile 19-23) und je nach Rückgabewert der neu erstelle Datenpunkt oder ein Fehler zurückgegeben.

    \begin{figure}[H]
        \centering
        \begin{minted}[linenos]{Rust}
async fn new_datapoint(
    State(AppState { database_pool, .. }): State<AppState>,
    Path(sensor_name): Path<String>,
    Json(new_datapoint): Json<NewSensorDatapoint>,
) -> ApiResult<SensorDatapoint, String> {
    // Herstellen einer Datenbank Verbindung
    let mut database_connection = database_pool.get().unwrap();

    // Überprüfung: Existiert das Device und ist es vom Typ Sensor?
    check_device_sensor_type(&mut database_connection, &sensor_name)?;

    // Erstellen des eigentlichen Datapoint-structs
    let sensor_datapoint = SensorDatapoint {
        timestamp: Utc::now(),
        value: new_datapoint.value
    };

    // Aufruf der zuvor beschriebenen Funktion der Datenbankschicht
    match database::sensor_datapoint::create_datapoint(
        &mut database_connection,
        sensor_datapoint,
        sensor_name,
    ) {
        // Falls Datenbank-Aufruf erfolgreich:
        Ok(datapoint) => Ok(
            ApiSuccess::new_with_status(
                StatusCode::OK,
                datapoint
            )),
        // Falls zu dem aktuellen Zeitpunkt schon ein Datenpunt existiert:
        Err(DatabaseError::UniqueKeyViolation) => Err(
            ApiError::new_with_status(
                StatusCode::CONFLICT,
                "Failed to create new datapoint.
                Wait a few seconds before trying again.".to_owned(),
            )),
        // Sonstiger Fehler:
        Err(err) => Err(ApiError::new_unknown(format!("{err:?}"))),
    }
}
        \end{minted}
        \caption{Rust-Funktion der Serviceschicht, welche den Endpunkt zum Erstellen eines Datenpunktes bereitstellt.}
        \label{lst:endpoint-code}
    \end{figure}

    Es ist außerdem wichtig zu erwähnen, dass durch erfolgte Abstraktion zum Teil unsichtbare Programmlogik stattfindet.
    Wird bspw. wie in Zeile 38 ein unbekannter \lstinline{ApiError} erstellt, wird der Fehler geloggt und der HTTP-Statuscode 500 "Internal Server Error" zurückgegeben.
    Das gleiche passiert auch bei jeglichen Rust-Panics, welche sich ähnlich wie Exceptions verhalten.
    So lässt sich der Code ohne Wiederholungen auf die wirklich nötige Programmlogik reduzieren.

\pagebreak
