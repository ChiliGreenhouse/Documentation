\section{Zusammenfassung und Reflexion}
Zusammenfassend ziehen wir eine sehr positive Bilanz aus dem Projekt. Wir haben im Laufe der Zeit viel über Softwareentwicklung und die Herausforderungen bei der Organisation dieser Entwicklung gelernt. Außerdem konnten wir einige unserer Ziele erreichen. Dazu gehört User Story 3, die Anzeige von Messdaten in Echtzeit. Die Messdaten werden automatisch ausgelesen und können überall auf einer Webseite eingesehen werden. Auf der Softwareseite ist es auch möglich, das Gewächshaus fernzusteuern (User Story 4). Dazu können über die Webseite Parameter eingestellt werden, die über das Backend an die Hardware übertragen werden. \\
Einige Ziele haben wir jedoch noch nicht erreicht. Wir haben noch nicht alle Sensoren und Aktoren final angeschlossen. So ist es softwareseitig schon möglich das Licht und die Bodenbewässerung anzusteuern, diese sind aber noch nicht eingebaut. Somit sind die User Stroy 1 und 2 noch nicht Final erreicht, können aber durch den Anschluss der Hardware vervollständigt werden. Dass nicht alle Ziele erreicht wurden, lag auch daran, dass nicht immer alles klappte oder auch mal falsch geplant wurde. Am Anfang haben wir gemeinsam die API Funktionalitäten besprochen und grob ausgearbeitet, jedoch waren diese noch nicht komplett ausgearbeitet, sodass im späteren Verlauf des Projektes die Teams aufeinander warten mussten, um an bestimmten Stellen weiterarbeiten zu können. Des Weiteren wurden einige wöchentliche Meetings aufgrund von Terminschwierigkeiten ausgesetzt und nicht nachgeholt, was zur Folge hatte, dass in einigen Wochen wenig bis gar kein Fortschritt erzielt wurde.\\
Wenn die Meetings stattfanden, waren sie durchaus produktiv. Innerhalb und zwischen den Teams wurde dann gut kommuniziert, wer was zu tun hat, was noch benötigt wird und welche neuen Aufgaben für die nächste Woche anstehen.\\
\\
Weitere Ideen für was gutes
\\


Insgesamt betrachten wir unser Projekt als erfolgreich. Wir haben es geschafft, unser Projekt zu planen, die Aufgaben gut zu verteilen und Ergebnisse zu erzielen. Es gab einige Fehler, die dazu geführt haben, dass wir nicht so viel erreicht haben, wie wir uns vorgenommen hatten, aber im Nachhinein haben wir diese Fehler erkannt und können sie in Zukunft vermeiden. Da das Projekt noch nicht abgeschlossen ist, können wir in Zukunft weiter daran arbeiten. So können wir sicherstellen, dass alle unsere Ziele erreicht werden. Außerdem ist während der Arbeit die Idee eines Systems mit mehreren Gewächshäusern entstanden, wofür auch schon der Grundstein gelegt wurde. Für die Zukunft kann das Projekt also noch erweitert werden.





\pagebreak
